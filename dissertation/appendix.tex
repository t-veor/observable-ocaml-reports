\appendix

\chapter{Implementation and Evaluation Addenda}

This appendix chapter provides some more details about finer details about the
implementation and the evaluation that could not fit within the main body of the
dissertation.

\section{The \texttt{Ccode} module} \label{ccode}

The \texttt{Ccode} module stores the OCaml representation of the target C
subset. It is specially chosen to be a subset of C, and only contain the
necessary types to represent the compilation target. There are four main
datatypes in the module, and a quick summary of them is given here:

\begin{itemize}

\item \texttt{Ccode.cident}, for representing identifiers (variable names, macro
    names, function names etc.) in C. This is largely a wrapper over the
    \texttt{Ident.t} type the OCaml compiler uses to represent identifiers,
    which is the string name along with a unique integer ID to disambiguate it
    from other identifiers with the same name.

\item \texttt{Ccode.cexpr}, for representing expressions in C. These include a
    wrapper around \texttt{cident}s, literals, and operations (such as unary
    operations, binary operations, function calls etc.) on other
    \texttt{cexpr}s.

\item \texttt{Ccode.cstatement}, for representing statements in C. Most data
    constructors in this type take a \texttt{cexpr} or a block of other
    \texttt{cstatement}s, and include if statements, while statements, switch
    statements etc. Notably a \texttt{cexpr} can be promoted to a
    \texttt{cstatement}, but not the other way around.

\item \texttt{Ccode.ctype}, for representing types in C. This is a auxiliary
    type used by the casting operator in \texttt{cexpr} and the variable
    declaration statement in \texttt{cstatement}. A more in-depth discussion of
    the types used to represent OCaml values is at \S\ref{value-repr}.

\end{itemize}

In addition, the \texttt{Ccode} module contains some special record types:

\begin{itemize}

\item \texttt{Ccode.cfunc}, which is a record type holding information about the
    return type of the function, the names and types of its arguments, its name,
    its body of code, and its location in the source.

\item \texttt{Ccode.ccode}, which is the record struct that is finally passed
    into \texttt{Cprint} to print the code. It is composed of three lists: the
    block of statements that forms the preamble (which is things like
    \texttt{\#}\texttt{include} directives and global variable declarations),
    the list of functions, and the block of code representing the \texttt{main}
    function.

\end{itemize}

\section{Purpose of \texttt{runtime.h}}

\texttt{runtime.h} is a special header file which needs to be included in the
compilation of the output C file to produce executables from my compiler. It
contains the type declarations mentioned in the implementation chapter, as well
as helper functions and macros used for implementing certain structures and
observability features.

\section{Compilation of basic constructs} \label{basic-constructs}

This section gives an overview of other common basic syntactical structures
in OCaml are compiled into C.

\subsection{Variable scoping differences between C and OCaml}
\label{variable-scoping}

In OCaml, each variable is only visible in the body of the expression where it
is bound, whether if that's in a let-binding or a function declaration. This is
in contrast to variable scope in C, where a variable is visible for the entirety
of the rest of the block it is in.

Furthermore, while OCaml does not allow reassignments (of variables, not 
references), it does allow you to bind another variable with the same name:

\begin{lstlisting}[language=Caml]
let x = 1 in
let x = 2 in
print_int x
\end{lstlisting} 

Here the second binding is not reassigning the value of \texttt{x}, rather it's 
creating another variable with the same name. The first variable still exists, 
but it's been shadowed by the second variable making it inaccessible. Note that 
after leaving the body of the second let binding the second variable goes out 
of scope and the first variable is accessible again.

This behaviour can be approximated using C99's block-scoping:

\begin{lstlisting}[language=C]
int x = 1;
{
    int x = 2;
    print_int(x);
}
\end{lstlisting}

Here we see that within the inner block, the second declaration of \texttt{x}
has shadowed the first declaration of \texttt{x}, but the original declaration
of \texttt{x} is still accessible once we leave the inner block (which is also
the behaviour in OCaml). Therefore, we have to pay careful attention to
inserting variable scopes to correctly mirror the shadowing effects in
OCaml\footnote{Note however we cannot apply this process for variable
    declarations at the top-level, since you cannot insert block-scopes at the
    top level in C. Unfortunately this requires top-level variables to be
renamed.}.

\subsection{Meta-notation for compilation} \label{meta-notation}

To describe the compilation process from OCaml expressions into C we use a
C-like meta-notation. Recall from \S\ref{expr-stmt}, each OCaml expression
compiles to a set of C statements and a variable, which the C statements will
have assigned the correct value to after their execution. We therefore denote
the set of C statements an expression \textit{\texttt{e}} compiles into as
\texttt{comp<\textit{e}>}, and the variable it is associated with with
\texttt{var<\textit{e}>}. This gives a way of recursively defining the
compilation process by defining \texttt{comp<\textit{e}>} for all possible OCaml
expressions. In the actual compiler, the \texttt{comp<>} functions are
implemented within the \texttt{Ccompile} module, which outputs an equivalent
representation of C using the types in the \texttt{Ccode} module.

As an example, consider the OCaml expression \texttt{\textit{e1} + \textit{e2}}.
We therefore define:

\begin{lstlisting}
comp<$\textit{e1}$ + $\textit{e2}$> :=

declare$\footnotemark$ result;
comp<$\textit{e1}$>; // var<$\textit{e1}$> now contains the correct value
comp<$\textit{e2}$>; // var<$\textit{e2}$> now contains the correct value
result = var<$\textit{e1}$> + var<$\textit{e2}$>;
\end{lstlisting}
\footnotetext{\texttt{declare} here stands for the type of the result of the
expression which is required for variable declarations in C. In general, this
type is not known generally and depends on the specific types of the
subexpressions, so we use \texttt{declare} to stand in for this.}

\texttt{result} refers to some temporary variable name which we assign to this
expression, and we define \texttt{var<\textit{e1} + \textit{e2}> := result}.
Note that this is a common pattern so we will implicitly assume that when
defining the compilation of an expression \texttt{\textit{e}},
\texttt{var<\textit{e}>} will be the variable named \texttt{result} within
\texttt{comp<\textit{e}>}.

\subsection{\texttt{if-then-else} expressions}

OCaml does not have if-statements but if-expressions, which are of the form:

\begin{center}
    \texttt{if \emph{cond} then \emph{A} else \emph{B}}
\end{center}

Using the same notational conventions as in \S\ref{meta-notation}:

\begin{lstlisting}
comp<if $\emph{cond}$ then $\emph{A}$ else $\emph{B}$> :=

declare result;

comp<$\emph{cond}$>;
if (var<$\emph{cond}$>) {
    comp<$\emph{A}$>;
    result = var<$\emph{A}$>;
} else {
    comp<$\emph{B}$>;
    result = var<$\emph{B}$>;
}
\end{lstlisting}

The same trick to propagate results outwards from inner scopes has been 
employed here also.

\subsection{\texttt{while} loops}

While loops are fairly simple in OCaml -- they simply repeatedly evaluate their 
body while the condition evaluates to true. There are no break nor continue 
statements, so there is not a concept of breaking out of a loop early with the 
exception of setting the condition to false. They have the syntax:

\begin{center}
    \texttt{while \emph{cond} do \emph{body} done}
\end{center}

In addition, the overall return value of the loop is \texttt{unit}. The
compilation for them is therefore: 

\begin{lstlisting}
comp<while $\emph{cond}$ do $\emph{body}$ done> :=

comp<$\emph{cond}$>;
declare temp = var<$\emph{cond}$>;
while (temp) {
    comp<$\emph{body}$>;
    comp<$\emph{cond}$>;
}

declare result = make_unit();
\end{lstlisting}

There is one interesting thing of note here: \texttt{\emph{cond}} does need to 
be evaluated twice, once before the loop, and once at the end of the loop. This 
is because the OCaml while loop re-evaluates \texttt{\emph{cond}} at the 
beginning of each iteration, but since OCaml expressions turn into a list of 
statements in C, we cannot fit this re-evaluation into the head of the while 
statement -- instead, a simple and equivalent way around this is to simply 
re-evaluate \texttt{\emph{cond}} at the end of the loop.

\subsection{\texttt{for} loops}

For loops in OCaml are extremely limited. They permit only iteration over a 
fixed range of integers, and also only allow increments in steps of 1. Like 
with while loops, there are no break nor continue statements in OCaml and so 
there is no possibility of exiting a loop early. They have two forms, which are:

\begin{center}
\texttt{for \emph{x} = \emph{start} to \emph{end} do \emph{body} done}\\
\texttt{for \emph{x} = \emph{start} downto \emph{end} do \emph{body} done}\\
\end{center}

These two versions simply iterate up to or down to a certain number 
(inclusive). Since for loops are so incredibly limited, it was decided 
that rather than including for loops in the targeted subset of C, it would be 
easier to also compile these into while loops in C.

\begin{lstlisting}
comp<for $\emph{x}$ = $\emph{start}$ to $\emph{end}$ do $\emph{body}$ done> :=

comp<$\emph{start}$>;
comp<$\emph{end}$>;
int temp1 = var<$\emph{start}$>;
int temp2 = var<$\emph{end}$>;

{
    decl x = temp1;
    while (x <= temp2) {
        comp<$\emph{body}$>;
        x++;
    }
}

decl result = make_unit();
\end{lstlisting}

(Replace \texttt{<=} for \texttt{>=} and \texttt{x++} for \texttt{x--} for 
\texttt{downto}.)

There are a few things about this compilation that warrant elaboration:
\begin{itemize}

\item \texttt{\emph{x}} again needs to be declared in its own scope, as the for
    loop acts as another variable binding site. Thus, the entirety of the for
    loop is wrapped in another block.

\item Note the use of \texttt{temp1} and \texttt{temp2}, which are necessary
    because \texttt{var<\emph{start}>} and \texttt{var<\emph{end}>} may collide
    with the name of the variable \texttt{x} and cause it to be shadowed in the
    inner scope.

\item \texttt{\emph{start}} and \texttt{\emph{end}} are only evaluated once at
    the start of the loop -- after some quick experiments it could be shown that
    OCaml does this as well, i.e. the limits of the iteration range do not
    change while the loop is running.

\item \texttt{\emph{x}} is mutated through the iteration of the loop. This is
    fine, as OCaml for loops only operate on integers, which are copied instead
    of referenced by other constructs in the target subset of C; thus the
    mutation of \texttt{\emph{x}} cannot affect the behaviour of the program.

\end{itemize}

\section{\texttt{Lstaticraise} and \texttt{Lstaticcatch}}

The \texttt{Lambda} IR also has support for static exceptions, which are
exceptions that can only occur across a local scope\footnote{In contrast to
normal exceptions, which unwind the stack and can jump to an error handler in an
enclosing scope.}. This is similar to the behaviour of a \texttt{goto} statement
in C. In particular, the OCaml pattern-match compiler may opt to use a static
exception to avoid duplication of code for certain branches. For example,
consider this match expression:

\begin{lstlisting}[language=Caml]
type color =
  | Red
  | Green
  | Blue
  | Gray of int
  | RGB of int * int * int

match c with
  | Red -> 0
  | Green -> 1
  | Gray x -> 2
  | _ -> 3
\end{lstlisting}

This match expression contains a wildcard expression which matches both 
\texttt{Blue}, which is an integer, and \texttt{RGB}, which is a block. The 
OCaml pattern-match compiler thus compiles it into the following 
\texttt{Lambda} expression:

\begin{lstlisting}
(catch
  (switch* c/xxxx
    case int 0: 0
    case int 1: 1
    case int 2: (exit 1)
    case tag 0: 2
    case tag 1: (exit 1))
  with (1) 3)
\end{lstlisting}

The branches for \texttt{Blue} and \texttt{RGB} have been compiled into a static
exception (the \texttt{exit} expression), to avoid duplicating the expression in
the branch. This behaviour can be emulated with \texttt{goto}s in C.
\texttt{Lstaticcatch} expressions can be compiled simply into an \texttt{if(0)}
and a label, and \texttt{Lstaticraise} expressions are simply compiled into a
\texttt{goto}.

\section{Functions that return other functions} \label{incomplete-funcs}

\S\ref{function-typing} provides a scheme for converting between function types
in OCaml to function types in C, but this conversion doesn't always make sense.
In OCaml, it is possible and often useful to define functions that return other
functions. As a simple example, consider the function:

\begin{lstlisting}[language=Caml]
let f x = (fun y -> x + y)
\end{lstlisting}

The type of \texttt{f} is \texttt{int -> int -> int}, but it's obvious that this
isn't translatable into a C function \texttt{int f(int x, int y)} -- the body of
the function would return a closure, and not an integer.

Since the type of \texttt{f} is known, this is a rather simple problem to fix
via a technique known as eta-expansion, and can be done at the source code
level. Simply add extra parameters to match the number of parameters in the
type, and apply them to the old result of the function.

\begin{lstlisting}[language=Caml]
let f x y = (fun y -> x + y) y
\end{lstlisting}

\section{``Reduced allocations'' mode} \label{reduced-allocs}

In the evaluation we mention that since floating point numbers must be boxed to
be stored in OCaml blocks or polymorphic structures, this makes operations with
them incredibly slow.  It however is still necessary for observability, as
non-integral or pointer types must be tagged with their type within polymorphic
structures so that a debugger is able to infer their types. If we relax this
requirement however, since the OCaml runtime does not otherwise require data to
be tagged with their type, we can store floating point numbers (and other types,
such as closures and strings) directly within OCaml blocks without needing to
make a separate object on the heap.

\begin{figure}
    \centering
    \resizebox{\textwidth}{!}{
        \includegraphics{figs/benchmarks-no-alloc.pdf}
    }

    \caption{Relative speed-up as before in figure~\ref{fig:raw-benchmarks}.
    While most of the benchmarks do not change much, note that the speed-up
    times for \texttt{mandelbrot} and \texttt{nbody} increase signficantly to
    more closely match the native compiler.}

    \label{fig:benchmarks-no-alloc}
\end{figure}

I implemented this mode which is accessible via a C compiler flag, and obtained
results which demonstrated far more favourable execution times especially for
the \texttt{mandelbrot} and \texttt{nbody} benchmarks, as can be seen in
figure~\ref{fig:benchmarks-no-alloc}.

While this change does increase the performance of the code generated by the
compiler, it is worth noting that this means that polymorphic structures are no
longer observable as a debugger cannot determine all the types at runtime, as
well as meaning that functions that require some degree of runtime reflection
such as OCaml's polymorphic comparisons do not function. This means that while
the results provided by this mode are interesting and represent the performance
of the compiler under perhaps a better representation of polymorphic data, the
results represent an interesting side scenario that is not suitable for use
generally. This is possibly fixable with the addition of further run-time
support for polymorphic structures.

\section{Stack overflow} \label{stack-overflow}

Curiously, from the benchmark results in figure~\ref{fig:raw-benchmarks}, the
executable produced by GCC under all optimisation levels segfaults for the
benchmark \texttt{sieve}. This is in spite of \texttt{sieve} being written in
tail-recursive style, and \texttt{clang} being able to produce a working
executable.

I therefore performed an investigation using M. Godbolt's compiler
explorer\footnote{\url{https://godbolt.org/}}, and narrowed down the offending
code that causes this to the following: 

\begin{lstlisting}[language=C]
typedef union {
    void* ptr;
} val;

void* g(val);

val f(val x) {
    return {.ptr = g(x)};
}
\end{lstlisting}

When compiled with gcc under \texttt{-O3}, this produces the rather confusing
output:

\begin{lstlisting}[basicstyle=\ttfamily\footnotesize,
basewidth={.5em,.3em}, frame=single]
0000000000400590 <f>:
  400590:       48 83 ec 08             sub    rsp,0x8
  400594:       e8 57 ff ff ff          call   4004f0 <g>
  400599:       48 83 c4 08             add    rsp,0x8
  40059d:       c3                      ret    
  40059e:       66 90                   xchg   ax,ax
\end{lstlisting}



Somehow, the cast back into the union type prevents \texttt{gcc} from detecting 
the function call is in tail-call position in this specific case, despite the 
fact that the binary representations for the pointer and the union can be the
same and so no conversion would be required in assembler. This causes the
function to compile to a decrement of the stack pointer, a call to another
function, and then an immediate increment of the stack pointer again, when it
would've been equivalent to a \texttt{jmp} instruction to \texttt{g}.


\chapter{Debug sessions} \label{debug-sessions}

\section{The ``sum'' program}

\input{appendices/sum_prog.tex}

\section{GDB session of the ``sum'' program} \label{debug-gdb}

\input{appendices/gdb_listing.tex}

\section{Ocamldebug session of the ``sum'' program} \label{debug-ocaml}

\input{appendices/ocamldebug_listing.tex}

\chapter{Project Proposal}

\maketitle

\textbf{Supervisor:} Stephen Kell\\
\textbf{Director of Studies:} Cecilia Mascolo\\
\textbf{Project Overseers:} Hatice Gunes \& Robert Watson

The project has progressed mostly smoothly up until now. The core of the 
project, the compiler from OCaml to C, is largely complete with the ability to 
compile working executables from a range of OCaml programs.

The tasks completed so far include:
\begin{itemize}
    \item Setting up existing OCaml compiler code to compile individual files 
    into the Lambda IR, and obtaining said representation;
    \item Writing a \texttt{Typecollect} module to obtain variable types from 
    the OCaml compiler, by iterating through the Lambda IR to find debug Lambda 
    events to obtain types;
    \item Creating a representation of a subset of C in OCaml, to which the 
    compiler will compile to;
    \item Translation of core language features of ML languages, including 
    functions, standard types (\texttt{int}, \texttt{float}, \texttt{string}, 
    \texttt{bool} and \texttt{unit}), simple expressions (\texttt{let}, 
    \texttt{let rec}, \texttt{if}, \texttt{for}, and \texttt{while}), referred 
    to as Subset 1 in the Proposal;
    \item Design of a representation for OCaml data structures, including 
    tagged unions, tuples, record types, lists, etc. and translation of 
    language features that use them (data constructors and pattern matching), 
    referred to as Subset 2 in the Proposal;
    \item Design of a closure representation to allow functions as first-class 
    values, and a closure conversion process to lift locally defined functions 
    and partially applied functions into closures which can be passed as 
    values, referred to as Subset 3 in the Proposal;
    \item Implementation of polymorphism in functions and data structures with 
    a generic union type, allowing functions to be more generic, also part of 
    Subset 3;
    \item Creation of a library of test OCaml programs to aid in regression 
    testing the compiler, as well as for the evaluation of compiled executable 
    speed and observability.
\end{itemize}

I am currently an estimated 1 -- 2 weeks behind the proposed schedule because 
of various difficulties encountered during the project, including:

\begin{itemize}
    \item The plan for 3 expanding subsets to be implemented sequentially was 
    quickly discovered to be impractical, because of the nature of how 
    different language features interact -- it would be difficult to attempt to 
    abstract different language features from each other before implementing 
    them together. While this did not represent a significant hindrance in 
    terms of time overall, it did mean that I couldn't produce tidy milestones 
    as in the Proposal, and it took longer before visible results could be 
    produced from the compiler.
    \item Originally, the \texttt{Typecollect} module looked at the Typed AST 
    of the code but it was quickly discovered that the Lambda IR introduces 
    extra untyped variables, so a workaround had to be developed to type these 
    the same way as polymorphic values until their type was more well known.
    \item A variety of minor kinks in how polymorphic values interact with 
    functions and closures, which has necessitated a few rewrites of how values 
    are represented.
\end{itemize}

Despite these difficulties no major setbacks have occurred so the intention is 
to carry on with the existing timetable, and I am confident that I can catch up 
with the work required.
