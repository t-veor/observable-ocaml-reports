\section*{Introduction and Description of the Work}

This project involves the implementation of a compiler from OCaml to C in such
a way that will allow the resulting executable to be compatible with
DWARF-based debugging tools, such as \texttt{gdb}. This project will use the
existing front-end of the OCaml compiler, but implement a very different
back-end which outputs valid C code that can be compiled using a tool such as
\texttt{gcc} into a debuggable executable. The focus of the compiler is to
optimise for debuggability, or ``observability'', by taking advantage of the
numerous debugging tools available of the C compilation toolchain.

Since the implementation of the full OCaml language is likely too ambitious for
one project, the core project will instead revolve around the implementation of
an interesting subset of OCaml which still provides technically interesting
challenges in its implementation, while extension tasks may extend the
implementation to cover a wider subset of the entire language. An informal
description of said subset will be given in the Substance and Structure
section.

The ``observability'' of the compiler means that where possible the debugger
should behave \emph{as if} it was debugging the OCaml code. This means that the
debugger should consider sections of the compiled machine code to be mapped to
the appropriate lines of code in OCaml. Furthermore, locally bound variables
should be visible from the debugger when execution has reached the relevant
point, meaning that local variables in the C code should mirror the variables
in the OCaml code.

In addition, features such as closures and parametric polymorphism must be
implemented carefully since these features don't have any similar structures in
C for which they could've been mapped to, but also to maintain
``observability'' these must be ``transparent'' to the debugger (for example, a
polymorphic function of the type \verb!'a -> 'a! that has been instantiated as
the type \verb!int -> int! should have this quality be visible from the
debugger). These qualities can be implemented using user-defined commands for
\texttt{gdb}.

The resulting compiler can be evaluated in mainly two contexts. One of these is
a straightforward performance comparison, where a testbed of OCaml programs are
benchmarked across the OCaml native compiler, the OCaml bytecode compiler, and
this project. The other context is the ``observability'' of the compiler; where
programs are stopped at a certain point of their execution and inspected, to
see how much of the internal state of the stack is recoverable. This can be
made in comparison with OCaml's bytecode debugger, \texttt{ocamldebug}.

\section*{Resources Required}

No special resouces are requred that are not open source and freely available
for download.

\section*{Starting Point}

The project, at least initially, will be a fork of the OCaml compiler
obtainable at \url{github.com/ocaml/ocaml} which will take the same front-end
as the existing compiler, but will implement a new back-end for compilation
into C. The work may use other libraries such as \texttt{libffi} and
\texttt{liballocs} but no work of similar scope to the project.

\section*{Substance and Structure of the Project}

The core of the project will be devoted into the implementation of three
incrementally expanding subsets of OCaml. These subsets have been identified
such that each is sufficient to write nontrivial programs in, and features have
been grouped by relation to each other. In this manner work can be easily split
into 3 discrete blocks, each of which will support a new collection of
programs by their completion.

\subsection*{Subset 1}

Subset 1 will be a very simple language with only a limited number of types and
language constructs. The subset will contain only the basic boolean, integer,
floating point and \texttt{unit} types, basic string support (for
input/output), top level function declarations with \texttt{let} and
\texttt{let rec}, which only accept one argument (so as to sidestep the problem
of partial application and closures), \texttt{if}, \texttt{for},
\texttt{while}, \texttt{ref}, as well as basic arithmetic and boolean
operations.

These features were chosen because because they very easily make for a small
minimalist language, and all constructs more or less have direct analogues in C
meaning translation from AST to C code should be fairly easy. Nonetheless, this
small subset is sufficient to write simple programs.

\subsection*{Subset 2}

Subset 2 will focus on the implementation of types and polymorphism, and will
support tuples, lists, parametric polymorphism, algebraic data types, record
types, and \texttt{match} expressions.

The implementation of this subset will likely require the design of an object
representation for OCaml values. Tuples and record types map easily to structs
in C, while a tagged union can be used for OCaml's ``sum of products'' approach
to algebraic data types. On the first pass, it is quite likely that a naive
``tagged pointer'' type will be used for support for polymorphism, but
extension tasks may experiment with eliminating this need via some method.

The addition of algebraic data types greatly increases the scope of supported
programs, and allows for functions taking multiple arguments (via tuples), list
processing functions, and more complex data structures such as binary trees.

\subsection*{Subset 3}

Subset 3 will finally add treatment of functions as first-class values. This
will include lambdas, lexical closures, and partial application of functions.

The implementation of this subset will require some representation of closures.
While C does support function pointers, it has no lexical closures and as such
a representation where closures are function pointers with an extra
\texttt{void *} pointer that refers to data in the closure. As an extension
task, more performant ways of implementing closures may be investigated, such
as potentially dynamically generating entry points that push the required
arguments before jumping to the body of the function. 

The addition of these features will make higher-order functions representable,
and make available functions such as \texttt{map}, \texttt{filter}, function
composition etc. At this point subset 3 can be considered a fairly minimal but
full-featured functional language.

\subsection*{Evaluation}

Evaluation will require the creation of a small library of test programs. These
programs should be novel and exhibit the full range of language features
supported by the subsets.

Two main methods will then be used to evaluate the compiler. One is
benchmarking performance of the test programs with the existing OCaml native
compiler and the OCaml bytecode compiler. While it is not expected that the C
compiler will match the performance of OCaml official compilers, the goal is to
be within a reasonable margin so that the performance of compiled C code is
comparable.

The other method will be the evaluation of the ``observability'' of the C code.
This will take place in two parts: firstly, we can do a test for
``observability'': this will entail a minimum list of requirements, such as
ability to step through OCaml code, ability to print locally bound values,
ability to print types of locally bound variables, and ability to show
instantiation of polymorphic functions into specific types. The second is to
compare the debug outputs of the C code and the OCaml code. The native OCaml
compiler does not allow observation of locally bound values at all, so our C
compiler should beat this easily; although a more interesting comparison would
be with the OCaml bytecode debugger. Breakpoints could be set at randomly
chosen call sites, and the stack traces, visible locals etc.\ could be compared
between \texttt{ocamldebug} and the debug output of the C code.

\subsection*{Extension Tasks}

This project leaves a lot of room for extension tasks. An obvious extension is
to extend the subset to be a more complete subset of OCaml. The next steps
taken would likely be module and exception support, although both are far more
complex features to attempt for implementation in C.

More realistic extension tasks would be changes to the existing code in order
to make it more performant. This would involve improvements to the
implementation of polymorphic functions and closures, as described in their
respective sections prior.

Another plausible extension task is to improve the debug output of OCaml
values. Since C does not have an exact analogue for all OCaml values the debug
output from \texttt{gdb} for example will not look exactly like their OCaml
representation. However, \texttt{gdb} supports the use of custom formatters
written in Python for types and a task to implement a way to generate
formatters to print OCaml values correctly is a suitable extension task.

\section*{Success Criteria}

The following criteria should be achieved at the completion of the project:

\begin{itemize}

    \item A working compiler, that is able to compile the language as described
        in Subset 3 into C code correctly;

    \item Comparable performance between the output of the compiler and the
        OCaml native and bytecode compilers;

    \item ``Observability'' in the compiled executable achieved, with tools
        such as \texttt{gdb} providing a coherent OCaml view of the execution
        of the program.

        This will involve the following:

        \begin{itemize}

            \item Ability to set breakpoints and step through the compiled executable
                as if it was OCaml code;

            \item Ability to view locally bound variables, and their types;

            \item Ability to print more complex data structures, such as ADTs,
                record types and lists;

            \item Ability to view the instantiated type of a polymorphic
                function;

            \item Similar and comparable output between the debug output and
                the same code running in \texttt{ocamldebug}.

        \end{itemize}

\end{itemize}

\section*{Timetable and Milestones}

\subsection*{20 October --- 3 November}

Initial experiments and reading. Familiarise with the OCaml compiler source
code and obtain typed AST\@. Set up development environment. Write testbed of
evaluation programs so that features can be tested and evaluated immediately
after their implementation.

Milestones: A completed small library of OCaml programs, plus a way of
obtaining the typed AST from the OCaml compiler.

\subsection*{4 November --- 17 November}

Implementation of Subset 1. Evaluation of Subset 1 performance with respect to
the OCaml native and bytecode compilers.

Milestones: A collection of programs now compileable by the Subset 1 compiler,
along with their running times collected in comparison with the OCaml
compilers.

\subsection*{18 November --- 15 December}

Implementation of Subset 2. This is expected to take longer than the
implementation of Subset 1 to take into account that special care needs to be
taken when designing object representations of OCaml values and implementation
of parametric polymorphism. Evaluation of Subset 2 performance with respect to
OCaml compilers.

Milestones: A larger collection of programs now compileable by the Subset 2
compiler, along with their performance results in comparison with the OCaml
compilers.

\subsection*{16 December --- 29 December}

(Two week break for Christmas)

\subsection*{30 December --- 12 January}

Implementation of Subset 3. Evaluation of Subset 3 performance, as well as
comparison between the debug output of the compiled code with
\texttt{ocamldebug}.

Milestones: Completion of core project with the implementation of Subset 3.
Some evaluation results (although perhaps not complete) for the core project.

\subsection*{13 January --- 26 January}

Write up of the Progress Report. Perform any new evaluation tasks that were
missed but become apparent in the write up of the Progress Report. Review the
remainder of the project in context with the existing work and determine which
extension tasks are best to pursue.

Milestones: Completed core project with evaluation, with a completed Progress
Report ready for submission. Entire project reviewed with supervisor and
overseers.

\subsection*{27 January --- 9 February}

Submission of Progress Report. Revise evaluation tasks with feedback from
Progress Report. Prepare Progress Report presentation.

\subsection*{10 February --- 23 February}

Start work on dissertation, as well as the implementation of ways of making the
compiled code more performant. Investigation and use of perhaps \texttt{libffi}
and \texttt{liballocs} to improve performance and observability.

Milestones: The introduction and preparation sections of the dissertation
complete. Improvements on the compiler made, with evaluation results to show
improvements in performance of the compiled output.

\subsection*{24 February --- 9 March}

Start writing the implementation section of dissertation. Implementation of the
generation of custom formatters for \texttt{gdb}, which will make the debug
output easier to read and interpret as OCaml values.

Milestones: Implementation section of the dissertation at least half complete.
Improvements to debug output of debuggers run on the compiled executables.

\subsection*{10 March --- 22 March}

Finish writing implementation section of dissertation. Start evaluation
section. Pick up any further evaluation tasks at this point that become
apparent.

Milestones: Finished implementation section, evaluation section at least half
complete. Revised evaluation tasks in accordance with the write up of the
evaluation section.

\subsection*{23 March --- 5 April}

Finish evaluation and conclusion sections of the dissertation. Deliver draft
dissertation to supervisor and director of studies for feedback. Spend time
working on whatever is in greatest need of attention.

Milestones: Submit a complete draft of the dissertation for review to
supervisor and director of studies.

\subsection*{6 April --- 19 April}

Revise dissertation based on feedback received. Finish any last tasks that
require performing, which are hopefully minor at this point in time. Send
revised drafts for further rounds of review.

\subsection*{20 April ---}

Final adjustments and revisions to dissertation and the body of code. Prepare
for submission of dissertation.

Milestone: Submission of dissertation.
