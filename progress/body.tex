\maketitle

\textbf{Supervisor:} Stephen Kell\\
\textbf{Director of Studies:} Cecilia Mascolo\\
\textbf{Project Overseers:} Hatice Gunes \& Robert Watson

The project has progressed mostly smoothly up until now. The core of the 
project, the compiler from OCaml to C, is largely complete with the ability to 
compile working executables from a range of OCaml programs.

The tasks completed so far include:
\begin{itemize}
    \item Setting up existing OCaml compiler code to compile individual files 
    into the Lambda IR, and obtaining said representation;
    \item Writing a \texttt{Typecollect} module to obtain variable types from 
    the OCaml compiler, by iterating through the Lambda IR to find debug Lambda 
    events to obtain types;
    \item Creating a representation of a subset of C in OCaml, to which the 
    compiler will compile to;
    \item Translation of core language features of ML languages, including 
    functions, standard types (\texttt{int}, \texttt{float}, \texttt{string}, 
    \texttt{bool} and \texttt{unit}), simple expressions (\texttt{let}, 
    \texttt{let rec}, \texttt{if}, \texttt{for}, and \texttt{while}), referred 
    to as Subset 1 in the Proposal;
    \item Design of a representation for OCaml data structures, including 
    tagged unions, tuples, record types, lists, etc. and translation of 
    language features that use them (data constructors and pattern matching), 
    referred to as Subset 2 in the Proposal;
    \item Design of a closure representation to allow functions as first-class 
    values, and a closure conversion process to lift locally defined functions 
    and partially applied functions into closures which can be passed as 
    values, referred to as Subset 3 in the Proposal;
    \item Implementation of polymorphism in functions and data structures with 
    a generic union type, allowing functions to be more generic, also part of 
    Subset 3;
    \item Creation of a library of test OCaml programs to aid in regression 
    testing the compiler, as well as for the evaluation of compiled executable 
    speed and observability.
\end{itemize}

I am currently an estimated 1 -- 2 weeks behind the proposed schedule because 
of various difficulties encountered during the project, including:

\begin{itemize}
    \item The plan for 3 expanding subsets to be implemented sequentially was 
    quickly discovered to be impractical, because of the nature of how 
    different language features interact -- it would be difficult to attempt to 
    abstract different language features from each other before implementing 
    them together. While this did not represent a significant hindrance in 
    terms of time overall, it did mean that I couldn't produce tidy milestones 
    as in the Proposal, and it took longer before visible results could be 
    produced from the compiler.
    \item Originally, the \texttt{Typecollect} module looked at the Typed AST 
    of the code but it was quickly discovered that the Lambda IR introduces 
    extra untyped variables, so a workaround had to be developed to type these 
    the same way as polymorphic values until their type was more well known.
    \item A variety of minor kinks in how polymorphic values interact with 
    functions and closures, which has necessitated a few rewrites of how values 
    are represented.
\end{itemize}

Despite these difficulties no major setbacks have occurred so the intention is 
to carry on with the existing timetable, and I am confident that I can catch up 
with the work required.